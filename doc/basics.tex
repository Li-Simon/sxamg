
\chapter{Basics}

This chapter introduces basic types, matrix and vector management.

\section{Data Types}

\verb|SXAMG| has two data types, \verb|SX_FLOAT| and \verb|SX_INT|, for floating-point number
and integer. They are defined as:
\begin{evb}
#if USE_LONG_DOUBLE
typedef long double              SX_FLOAT;
#else
typedef double                   SX_FLOAT;
#endif

#if USE_LONG_LONG
typedef signed long long int     SX_INT;
#elif USE_LONG
typedef signed long int          SX_INT;
#else
typedef signed int               SX_INT;
#endif

\end{evb}

The macros, \nverb{USE_LONG_DOUBLE}, \nverb{USE_LONG_LONG} and \nverb{USE_LONG}, are set by \nverb{configure} to
control the real types. User can control through  \nverb{configure} options.

\section{Matrix}
\nverb{SX_MAT} defines CSR matrix. The index is from zero. The meanings of the members are clear.
\begin{evb}
typedef struct SX_MAT
{
    SX_INT num_rows;
    SX_INT num_cols;
    SX_INT num_nnzs;

    SX_INT   *Ap;
    SX_INT   *Aj;
    SX_FLOAT *Ax;

} SX_MAT;
\end{evb}

\subsection{Matrix Management}

\subsubsection{Create}

\nverb{sx_mat_struct_create} creates the structure of a matrix, and no value is set:
\begin{evb}
SX_MAT sx_mat_struct_create(const SX_INT nrow, const SX_INT ncol, const SX_INT nnz);
\end{evb}

\nverb{sx_mat_create} creates a CSR matrix:
\begin{evb}
SX_MAT sx_mat_create(SX_INT nrow, SX_INT ncol, SX_INT *Ap, SX_INT *Aj, SX_FLOAT *Ax);
\end{evb}

\subsubsection{Destroy}
\nverb{sx_mat_destroy} destroys a matrix and frees its memory:
\begin{evb}
void sx_mat_destroy(SX_MAT *A);
\end{evb}

\subsubsection{Transpose}
\nverb{sx_mat_trans} gets transpose a matrix:
\begin{evb}
SX_MAT sx_mat_trans(SX_MAT *A);
\end{evb}

\subsubsection{Sort}
\nverb{sx_mat_sort} sorts column indices in ascending manner.
\begin{evb}
void sx_mat_sort(SX_MAT *A);
\end{evb}

\section{Vector}

\nverb{SX_VEC} defines floating-point vector. $n$ is the length of the vector, and $d$
stores the values.
\begin{evb}
typedef struct SX_VEC 
{
    SX_INT   n;
    SX_FLOAT *d;

} SX_VEC;
\end{evb}

\subsection{Vector Management}

\subsubsection{Create}
\nverb{sx_vec_create} creates a vector of length $m$.
\begin{evb}
SX_VEC sx_vec_create(SX_INT m);
\end{evb}

\subsubsection{Destroy}

\nverb{sx_vec_destroy} destroys a vector and frees its memory.
\begin{evb}
void sx_vec_destroy(SX_VEC *u);
\end{evb}

\subsubsection{Set Value}

\nverb{sx_vec_set_value} sets equal value to each component.
\begin{evb}
void sx_vec_set_value(SX_VEC *x, SX_FLOAT val);
\end{evb}

\nverb{sx_vec_set_entry} sets value: x[index] = val,
\begin{evb}
void sx_vec_set_entry(SX_VEC *x, SX_INT index, SX_FLOAT val);
\end{evb}

\subsubsection{Get Value}
\nverb{sx_vec_get_entry} returns value of x[index],
\begin{evb}
SX_FLOAT sx_vec_get_entry(SX_VEC *x, SX_INT index);
\end{evb}

\subsubsection{Copy}
\nverb{sx_vec_cp} copies source vector to destination vector.
\begin{evb}
void sx_vec_cp(SX_VEC *src, SX_VEC *des);
\end{evb}

\section{BLAS Operations}

\subsection{Vector}
\nverb{sx_blas_vec_norm2} calculates L2 norm.
\begin{evb}
SX_FLOAT sx_blas_vec_norm2(SX_VEC *x);
\end{evb}

\nverb{sx_blas_vec_dot} calculates dot product.
\begin{evb}
SX_FLOAT sx_blas_vec_dot(SX_VEC *x, SX_VEC *y);
\end{evb}

\nverb{sx_blas_vec_axpyz} computes: $z = a * x + y$.
\begin{evb}
void sx_blas_vec_axpyz(const SX_FLOAT a, SX_VEC *x, SX_VEC *y, SX_VEC *z);
\end{evb}

\nverb{sx_blas_vec_axpy} computes: $y = a * x + y$.
\begin{evb}
void sx_blas_vec_axpy(const SX_FLOAT a, SX_VEC *x, SX_VEC *y);
\end{evb}

\subsection{Matrix-Vector}
\nverb{sx_blas_mat_amxpy} computes: $ y = y + a * A * x $,
\begin{evb}
void sx_blas_mat_amxpy(const SX_FLOAT alpha, SX_MAT *A, SX_VEC *x, SX_VEC *y);
\end{evb}

\nverb{sx_blas_mat_mxy} computes: $y = A * x $,
\begin{evb}
void sx_blas_mat_mxy(SX_MAT *A, SX_VEC *x, SX_VEC *y);
\end{evb}

\subsection{Matrix-Matrix}
\nverb{sx_blas_mat_rap} returns the matrix-matrix product: $ R * A * P$,
\begin{evb}
SX_MAT sx_blas_mat_rap(SX_MAT *R, SX_MAT *A, SX_MAT *P);
\end{evb}
